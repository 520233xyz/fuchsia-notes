\section{设备模型和驱动}
有一个中心服务devmgr coordinator. 它会扫描驱动,为每个驱动launch一个devhost进程。launch完了之后会发送
create device, bind driver这2个rpc给devhost. devhost会处理这2个rpc. 对于root device,不会处理,
因为libname=""

文档推荐在bind driver里调用add device。驱动本身是在devhost里运行的。devhost会rpc调用dev coordinator
的add device方法。这个方法主要是创建dc里对设备的代表,发布到devfs里。


在启动进程的时候,传递的第一个参数是一个channel handle.
libc start main在拿到这个channel handle之后,读取它。channel里的message是MessagePack
里面的data是\verb|zx_proc_args_t|。这个东西是由另一头用\verb|channel_write()|写进来的。
userboot里面是手工组装这个消息。如果是普通应用层应该会把这个封装掉。

\verb|zx_proc_args_t|是传递进程参数的固定的消息格式。作为data写入MessagePack.


到达devmgr:main()
\begin{verbatim}
devfs_init():
    iostate_create(&root_devnode)
devmgr_vfs_init():
    fshost_start():
              loader_service:
svc async的port等待的是另外一头的handle
async里的线程等待的是port
dl_set_loader_service()   把新创建的loader service设置为缺省的,关闭老的

devfs_root_clone(): 创建一个channel, 返回一头,另一头封装在一个消息里发送给devfs_root
                                devfs_root会在coordinator()中被轮询
              问题在于devfs_root的另一头只是和dc_port   wait了一次。在处理完成之后
              会立刻又wait一次!
               另一头会传递给devfs_open()


fs_clone("svc"): 
        fdio_open_at(fs=appmgr_req_cli, path, flag, h1)
        this function creates a channel and send one handle to appmgr, and returns the other 
\end{verbatim}

设置了一堆handle之后,启动fshost      
use launchpad, to be looked ...
    
发送了几个设备的channel handle给\verb|fs_root|



