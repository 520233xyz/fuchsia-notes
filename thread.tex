\section{线程}

创建bootstrap2线程时,\verb|thread_t| 是在malloc里创建的。stack也是在malloc.
如果是用户进程通过系统调用创建线程,则stack是通过kernel aspece分配出来的vm object.
user sp是通过vm object mapping得到的。

放到stack frame里的入口是\verb|initial_thread_func|。这个函数会调用\verb|thread_t|的entry.
\verb|thread_t|的entry是ThreadDispatcher::StartRoutine. StartRoutine里会调用
\verb|user_entry_|,这个是在ThreadDispatcher::Start()里设置的。

thread的启动:ThreadDispatcher::Start()里设置好\verb|user_entry_|,然后触发线程调度,
切换到线程的stack上,然后进入stack frame里指定的入口\verb|initial_thread_func|.

thread的退出:在\verb|thread_create_etc()|里,线程入口被设置为\verb|initial_thread_func|。
在这个函数里,如果是用户线程,则不会回来了。用户线程自己调用\verb|process_exit|退出。
如果是kernel线程,会回到这里调用\verb|thread_exit()|。在这个函数里,线程把自己从调度队列里拿下来,然后
触发调度函数,进入别的线程。

中断:中断会保存\verb|spsr_el1|,然后切换到\verb|sp_el1|,然后进入中断向量