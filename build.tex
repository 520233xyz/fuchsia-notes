\section{Fuchsia Build}
gn args out/x64/ --list --short
这个命令显示所有的变量的预设值。

gn desc out/x64 //garnet/bin/appmgr:bin
描述一个target的全部信息。


fx set arm64 out/mydir

fx调用gn工具处理.gn文件,生成.ninja文件。ninja根据.ninja文件进行构建。

https://chromium.googlesource.com/chromium/src/tools/gn/

https://ninja-build.org/

\begin{verbatim}
/home/xzx/fuchsia/buildtools/gn gen /home/xzx/fuchsia/out/x64 
  --check '--args=target_cpu="x64" fuchsia_packages=["topaz/packages/default",]'

gn gen是生成ninja文件。
--args=设置一些变量供gn使用。
  target_cpu="x64" 
  fuchsia_packages=["topaz/packages/default",]

gn的缺省变量:
   - host_cpu
   - host_os
   - current_cpu
   - current_os
   - target_cpu
   - target_os

target_os = "fuchsia"
target_cpu="x64"
current_cpu = target_cpu
current_os = target_os

select_variant = []
select_variant_canonical = []
select_variant_shortcuts = [{
  select_variant = [{
  variant = "asan_no_detect_leaks"
  host = true
  dir = ["//third_party/yasm", "//third_party/vboot_reference", "//garnet/tools/vboot_reference"]
}, {
  variant = "asan"
  host = true
}]
  name = "host_asan"
}]

gn启动之后第一件事是寻找当前目录以及父目录里的.gn文件,确定源码树的根。从.gn文件开始执行。
.gn可能会定义secondary_source目录。gn也会去这里找文件。
寻找buildconfig文件,执行。
执行//BUILD.gn, fuchsia设置了root = "//build/gn", 所以会去这里找BUILD.gn
root  
  Label of the root build target. The GN build will start by loading the
  build file containing this target name. This defaults to "//:" which will
  cause the file //BUILD.gn to be loaded.

当一个target完整之后,输出ninja文件。
最后输出build.ninja


gn 预定义的变量:
target_out_dir
{{source_out_dir}}
      The object file directory (*) corresponding to the source file's path,
      relative to the build directory. this us be different than the target's
      out directory if the source file is in a different directory than the
      build.gn file.
        "//foo/bar/baz.txt" => "obj/foo/bar"

label包括target, config, toolchain

All targets encountered in the default toolchain (see "gn help toolchain")
will have build rules generated for them, even if no other targets reference
them.

buildconfig [required]
      Path to the build config file. This file will be used to set up the
      build file execution environment for each toolchain.

gn 定义变量:
var = value
host_toolchain = "//build/toolchain:host_x64"
target_toolchain = "//build/toolchain/fuchsia:x64"

toolchain_variant = {
  base = "//build/toolchain/fuchsia:x64"
}
toolchain_variant.name = ""
toolchain_variant.suffix = ""
toolchain_variant.configs = []
toolchain_variant.remove_common_configs = []
toolchain_variant.remove_shared_configs = []
toolchain_variant.deps = []
toolchain_variant.is_pic_default = false
is_linux = true
default_library_type = "static_library"

toolchain_variant = {
  base = "//build/toolchain/fuchsia:x64"
}
universal_variants = [{
  toolchain_args = {
  is_debug = false
}
  configs = []
  name = "release"
}]

clang_toolchain_suite("x64")
  is_host = false
只有下列variant:
{
  base = "//build/toolchain/fuchsia:x64"
  configs = []
  remove_shared_configs = []
  remove_common_configs = []
  suffix = ""
  deps = []
  name = ""
}
variant_used
{
  base = "//build/toolchain:host_x64"
  configs = []
  remove_shared_configs = []
  remove_common_configs = []
  suffix = ""
  deps = []
  name = ""
}

\end{verbatim}