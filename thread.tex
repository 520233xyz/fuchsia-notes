\section{Process and Thread}

创建bootstrap2线程时,\verb|thread_t| 是在malloc里创建的。stack也是在malloc.
如果是用户进程通过系统调用创建线程,则stack是通过kernel aspece分配出来的vm object.
user sp是通过vm object mapping得到的。

放到stack frame里的入口是\verb|initial_thread_func|。这个函数会调用\verb|thread_t|的entry.
\verb|thread_t|的entry是ThreadDispatcher::StartRoutine. StartRoutine里会调用
\verb|user_entry_|,这个是在ThreadDispatcher::Start()里设置的。

thread的启动:ThreadDispatcher::Start()里设置好\verb|user_entry_|,然后触发线程调度,
切换到线程的stack上,然后进入stack frame里指定的入口\verb|initial_thread_func|.

thread的退出:在\verb|thread_create_etc()|里,线程入口被设置为\verb|initial_thread_func|。
在这个函数里,如果是用户线程,则不会回来了。用户线程自己调用\verb|process_exit|退出。
如果是kernel线程,会回到这里调用\verb|thread_exit()|。在这个函数里,线程把自己从调度队列里拿下来,然后
触发调度函数,进入别的线程。

中断:中断会保存\verb|spsr_el1|,然后切换到\verb|sp_el1|,然后进入中断向量


在启动进程的时候,传递的第一个参数是一个channel handle.
libc start main在拿到这个channel handle之后,读取它。channel里的message是MessagePack
里面的data是\verb|zx_proc_args_t|。这个东西是由另一头用\verb|channel_write()|写进来的。
userboot里面是手工组装这个消息。如果是普通应用层应该会把这个封装掉。

\verb|zx_proc_args_t|是传递进程参数的固定的消息格式。作为data写入MessagePack.

\section{进入main()之前的操作}

应用程序的入口是main(),在此之前,libc会做一些设置。

\begin{verbatim}

launchpad_create(job_copy, name, &lp);
  launchpad_create_with_jobs(job, xjob, name, result);
launchpad_load_from_vmo(lp, svchost_vmo);
  launchpad_file_load_with_vdso(lp, vmo);
    launchpad_file_load(lp, vmo);
      先检查是不是#!脚本,如果不是:
      launchpad_elf_load_body(lp, first_line, to_read, vmo);
        elf_load_start(vmo, hdr_buf, buf_sz, &elf)
          elf_load_prepare(vmo, hdr_buf, buf_sz, &header, &phoff)
          读程序头
        elf_load_get_interp(elf, vmo, &interp, &interp_len);
          elf_load_find_interp(info->phdrs, info->header.e_phnum,
            寻找动态链接器
          handle_interp(lp, vmo, interp, interp_len)
            setup_loader_svc(lp);
            elf_load_finish(lp_vmar(lp), elf, interp_vmo,
                                 &segments_vmar, &lp->base, &lp->entry)
              把动态链接器的入口作为lp的入口


进程创建的时候创建root vmar

launchpad提供了一系列接口,包括:加载可执行文件的vmo,添加启动命令行参数,添加handle。

launchpad_go(launchpad_t* lp, zx_handle_t* proc, const char** errmsg)
  launchpad_start(lp, &h)
    prepare_start(lp, &data);
      zx_channel_create(0, &to_child, &bootstrap);
      zx_thread_create(lp_proc(lp), thread_name, strlen(thread_name), 0, &thread);
      给新进程创建stack vmo
    zx_process_start(data.process, data.thread, data.entry, data.sp,
                              data.bootstrap, data.vdso_base);
      In kernel space:
      thread->Start(pc, sp, static_cast<uintptr_t>(arg_nhv),
                              arg2, /* initial_thread */ true)              

动态链接库是libc.so. 入口在dl-entry.S文件里。这里会链接真正的可执行文件,找到
它的入口地址,进入它。实现在zircon/third_party/ulib/musl/ldso/dynlink.c

_dl_start(void* start_arg, void* vdso)

__dls3

可执行文件的最初的入口在zircon/third_party/ulib/musl/arch/aarch64/Scrt1.S
mov main@GOTPCREL(%rip), %rsi
jmp *__libc_start_main@GOTPCREL(%rip)

argument在rdi中,main在rsi中。

__libc_start_main(void* arg, int (*main)(int, char**, char**))
  bootstrap = (uintptr_t)arg;
  把handles取出来
  把names取出来
  设置一些全局handle
    __environ
    __zircon_process_self;
    __zircon_vmar_root_self;
    __zircon_job_default;


  start_main(const struct start_params* p)
    __libc_extensions_init(p->nhandles, p->handles, p->handle_info,
                               p->namec, p->names);
    __libc_startup_handles_init(p->nhandles, p->handles, p->handle_info);

    // Run static constructors et al.
    __libc_start_init();
                           
    // Pass control to the application.
    exit((*p->main)(p->argc, p->argv, __environ));  

\end{verbatim}