\section{acpi}

\begin{verbatim}
coordinator(void):
static device_t sys_device = {
    .parent = &root_device,
    .flags = DEV_CTX_IMMORTAL | DEV_CTX_MUST_ISOLATE,
    .name = "sys",
    .libname = "",
    .args = "sys,",
    .children = LIST_INITIAL_VALUE(sys_device.children),
    .pending = LIST_INITIAL_VALUE(sys_device.pending),
    .metadata = LIST_INITIAL_VALUE(sys_device.metadata),
    .refcount = 1,
};
sys_device.libname = "/boot/driver/bus-acpi.so";
dc_prepare_proxy(&sys_device);   
  dc_create_proxy(dev) 
    parent->proxy = dev;

  need_proxy_rpc = false

  h1 = bootdata_vmo;


  dc_new_devhost(devhostname, dev->host, &dev->proxy->host)
    dev->proxy->host 具备了与devhost进程通信的通道hrpc

  dh_create_device(dev->proxy, dev->proxy->host, arg1, h1)
    dev是sys_device, dev->proxy->host通向devhost, h1是bootdata_vmo, arg1指向dev->args逗号后面,这里是空
    handle[0]: rpc通道
    handle[1]: bus-acpi.so vmo
    handle[2]: bootdata vmo
    msg.op = DC_OP_CREATE_DEVICE;

    zx_channel_write(dh->hrpc, 0, &msg, mlen, handle, hcount)
      通过刚刚建立的rpc通道发送过去
    
    把新的rpc通道挂到dev->proxy上面

  
case DC_OP_ADD_DEVICE:
  dc_add_device(dev, hin[0], &msg, name, args, data,
                               msg.op == DC_OP_ADD_DEVICE_INVISIBLE))

case WORK_DEVICE_ADDED:
  dc_handle_new_device(dev);
    dc_attempt_bind(drv, dev);
      non-bus设备不用启动新的devhost   kpci是non bus设备
      dh_bind_driver(dev, drv->libname);
        引起devhost端的kpci.c:pci_init_child

----

devhost: device_host_main()
启动阶段:
devhost_io_init();
  fd: 1,2->log output

port_wait(&dh_port, &root_ios.ph)
  建立最初的通道:dev->proxy->host->hrpc  <=>   root_ios.ph.handle

port_dispatch(&dh_port, ZX_TIME_INFINITE, false)
  监听port

----
devhost:
case DC_OP_CREATE_DEVICE:
  
  dh_find_driver(name, hin[1], &drv)
  zx_device_t parent = {
    .name = "device_create dummy",
  };

  creation_context_t ctx = {
      .parent = &parent,
      .child = NULL,
      .rpc = hin[0],
  };
  devhost_set_creation_context(&ctx);
  r = drv->ops->create(drv->ctx, &parent, "proxy", args, hin[2]);   drv->ctx == null
  acpi_drv_create(void* ctx, zx_device_t* parent, const char* name,
                                   const char* _args, zx_handle_t zbi_vmo)
    init()
    install_powerbtn_handlers();
    pci_report_current_resources(get_root_resource());

    device_add_args_t args = {
        .version = DEVICE_ADD_ARGS_VERSION,
        .name = name,
        .ops = &sys_device_proto,
        .flags = DEVICE_ADD_NON_BINDABLE,
    };

    device_add(parent, &args, &sys_root);
      device_add_from_driver(__zircon_driver_rec__.driver, parent, args, out)
        devhost_device_create(drv, parent, args->name, args->ctx, args->ops, &dev);
        devhost_device_add(dev, parent, args->props, args->prop_count, NULL);
          dev->flags |= DEV_FLAG_ADDED;
          dev->flags &= (~DEV_FLAG_BUSY);
          dev->rpc = ctx->rpc;
          ctx->child = dev;
          return ZX_OK;

    device_add_args_t args2 = {
        .version = DEVICE_ADD_ARGS_VERSION,
        .name = "acpi",
        .ops = &acpi_root_device_proto,
        .flags = DEVICE_ADD_NON_BINDABLE,
    };

    acpi_root = NULL;
    device_add(sys_root, &args2, &acpi_root);
      device_add_from_driver(__zircon_driver_rec__.driver, parent, args, out)
        devhost_device_create(drv, parent, args->name, args->ctx, args->ops, &dev);
        devhost_device_add(dev, parent, args->props, args->prop_count, NULL);
          devhost_add(parent, dev, proxy_args, props, prop_count);
            msg.op = DC_OP_ADD_DEVICE;

            "acpi"设备创建了, protocol id == 0

    publish_acpi_devices(acpi_root);

      publish_acpi_device_ctx_t ctx = {
          .parent = parent,
          .found_pci = false,
          .last_pci = -1,
      };
      AcpiWalkNamespace(ACPI_TYPE_DEVICE,
                                                  ACPI_ROOT_OBJECT,
                                                  MAX_NAMESPACE_DEPTH,
                                                  acpi_ns_walk_callback,
                                                  NULL, &ctx, NULL);
        publish_device(parent, object, info, "pci",
                    ZX_PROTOCOL_PCIROOT, &pciroot_proto);
          device_add()
            "pci root"设备创建了
\end{verbatim}