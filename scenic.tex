\section{scenic}

\begin{verbatim}
service_root通向"/svc/"

CreateStaticServiceRootHandle()
    fdio_service_connect("/svc/.", h1.release())
      fdio_ns_connect(fdio_root_ns, svcpath,
                               ZX_FS_RIGHT_READABLE | ZX_FS_RIGHT_WRITABLE, h);

        __libc_extensions_init里面会把fdio的各种handle设置好。
        这样app就可以用fdio_ns_connect来获取channel了
        如果root namespace里面没有"/svc/",则这个进程不能访问这个名字

StartupContext::StartupContext(zx::channel service_root, zx::channel directory_request) {
  incoming_services_.Bind(std::move(service_root));
    directory_ = std::move(directory);
  
  outgoing_.Serve(std::move(directory_request));
    vfs_.ServeDirectory(root_dir_, std::move(dir_request))
  
  environment_.NewRequest()  // InterfacePtr<Environment>::NewRequest()
    zx::channel::create(0, &h1, &h2)
    Bind(std::move(h1), async)
      impl_->controller.reader().Bind(std::move(channel), async);  ????
    
    return InterfaceRequest<Interface>(std::move(h2))

    把通道的一头挂在本地事件循环里,Environment服务的客户端。


  incoming_services_.ConnectToService(environment_.NewRequest());


  incoming_services_.ConnectToService(launcher_.NewRequest());
}

App::App(fuchsia::sys::StartupContext* app_context, fit::closure quit_callback)
    : scenic_(std::make_unique<Scenic>(app_context, std::move(quit_callback)))

    app_context->outgoing().AddPublicService<fuchsia::ui::scenic::Scenic>(
        [this](fidl::InterfaceRequest<fuchsia::ui::scenic::Scenic> request) {
          FXL_VLOG(1) << "Accepting connection to Scenic";
          scenic_bindings_.AddBinding(this, std::move(request));
        });


    scenic_->RegisterSystem<scenic::gfx::GfxSystem>()     // SystemT = scenic::gfx::GfxSystem
        system = new GfxSystem(SystemContext(app_context_, quit_callback_.share())
            display_manager_.WaitForDefaultDisplay(cb1)
                display_watcher_.WaitForDisplay(cb2)
                    device_watcher_ = fsl::DeviceWatcher::Create("/dev/class/display-controller", cb3)
                        open("/dev/class/display-controller", O_DIRECTORY | O_RDONLY);
                        zx_channel_create(0, &wd.channel, &dir_watch_handle)
                        ioctl_vfs_watch_dir(dir_fd.get(), &wd)

如果/dev/class/display-controller存在,调用DeviceWatcher::Handler
DisplayWatcher::HandleDevice
        dc_fd_ = std::move(fd);
        dc_channel_ = dc_handle.get();
        display_controller_.Bind(std::move(dc_handle));

        // TODO(FIDL-183): Resolve this hack when synchronous interfaces
        // support events.
        auto dispatcher = static_cast<fuchsia::display::Controller::Proxy_*>(
            event_dispatcher_.get());
        dispatcher->DisplaysChanged = [this](auto added, auto removed) {
          DisplaysChanged(std::move(added), std::move(removed));
        };
        dispatcher->ClientOwnershipChange = [this](auto change) {
          ClientOwnershipChange(change);
        };

        wait_.set_object(dc_channel_);
        wait_.set_trigger(ZX_CHANNEL_READABLE | ZX_CHANNEL_PEER_CLOSED);
        wait_.Begin(async_get_default());

        display manager拿到dc_channel_


先看一下scene_manager_vulkan_swapchain=2的情况
Initialize()
    escher::VulkanIsSupported()
        CheckIfVulkanIsSupported()
            CreateVulkanInstance()
                vk::createInstance(&instance_info, nullptr, &instance)
                    public/lib/escher/third_party/vulkansdk/x86_64/include/vulkan/vulkan.hpp
                    ::vkCreateInstance( pCreateInfo, pAllocator, pInstance);
                        in file //garnet/public/lib/escher/third_party/vulkansdk/source/loader/trampoline.c




\end{verbatim}