\section{设备模型和驱动}
有一个中心服务devmgr coordinator. 它会扫描驱动,为每个驱动launch一个devhost进程。launch完了之后会发送
create device, bind driver这2个rpc给devhost. devhost会处理这2个rpc. 对于root device,不会处理,
因为libname=""

文档推荐在bind driver里调用add device。驱动本身是在devhost里运行的。devhost会rpc调用dev coordinator
的add device方法。这个方法主要是创建dc里对设备的代表,发布到devfs里。


在启动进程的时候,传递的第一个参数是一个channel handle.
libc start main在拿到这个channel handle之后,读取它。channel里的message是MessagePack
里面的data是\verb|zx_proc_args_t|。这个东西是由另一头用\verb|channel_write()|写进来的。
userboot里面是手工组装这个消息。如果是普通应用层应该会把这个封装掉。

\verb|zx_proc_args_t|是传递进程参数的固定的消息格式。作为data写入MessagePack.

\begin{verbatim}
dc_prepare_proxy()
        dc_create_proxy()
        dc_new_devhost()
        dh_create_device()
\end{verbatim}


到达devmgr:main()
\begin{verbatim}
devfs_init():
    iostate_create(&root_devnode)
devmgr_vfs_init():
    fshost_start():
              loader_service:
svc async的port等待的是另外一头的handle
async里的线程等待的是port
dl_set_loader_service()   把新创建的loader service设置为缺省的,关闭老的

devfs_root_clone(): 创建一个channel, 返回一头,另一头封装在一个消息里发送给devfs_root
                                devfs_root会在coordinator()中被轮询
              问题在于devfs_root的另一头只是和dc_port   wait了一次。在处理完成之后
              会立刻又wait一次!
               另一头会传递给devfs_open()


fs_clone("svc"): 
        fdio_open_at(fs=appmgr_req_cli, path, flag, h1)
        this function creates a channel and send one handle to appmgr, and returns the other 
\end{verbatim}

设置了一堆handle之后,启动fshost      
use launchpad, to be looked ...
    
发送了几个设备的channel handle给\verb|fs_root|



加载驱动的过程:

dc进程:
\begin{verbatim}
coordinator()
  find_loadable_drivers()
    di_read_driver_info()
      found_driver()
        dc_driver_added()
          if dc_running  then queue WORK_DRIVER_ADDED  实际上dc_running == false

process_work()
WORK_DRIVER_ADDED:
  dc_handle_new_driver()
    dc_bind_driver()
      dc_attempt_bind()
        dc_prepare_proxy()
          dc_create_proxy()  为父设备创建一个代理设备
          创建h0, h1
          dc_new_devhost()   proxy->devhost会与devhost进程有channel连接
          dh_create_device()  向devhost进程发送DC_OP_CREATE_DEVICE消息,
                              创建新的channel. 一头设置给proxy dev,在dc里监听起来。
                              另一头作为handle[0]传给devhost(存放在devhost里proxy device里面,
                              作为其他设备的parent).  从而dc和dh里面的proxy device之间
                              建立了rpc通道。


                              h1作为handle[2]传过去,这个目前似乎只有pci_rpc_request用于
                              发布出去的rpcch,
                              跟别的关系不大。

          dh_connect_proxy()  把h0作为DC_OP_CONNECT_PROXY传过去
        dh_bind_driver()  向dh进程发送DC_OP_BIND_DRIVER消息。

DC_OP_ADD_DEVICE:
  dc_add_device()
    创建子device, 把发来的hrpc设置好。监听。
    添加WORK_DEVICE_ADDED

WORK_DEVICE_ADDED:
  dc_handle_new_device()
    寻找匹配驱动
    dc_attempt_bind()
                
dh进程:
DC_OP_CREATE_DEVICE:
  dh_find_driver()
  driver->ops->create(hin[2])
    device_add()
      device_add_from_driver()
        devhost_device_create()
        devhost_device_add()
          proxy device case: 把rpc记录到dev里面,返回
          
  监听hin[0], 至此proxydev和 devhost之间建立了rpc通道。

DC_OP_CONNECT_PROXY:
  proxy_ios_create()  在devhost进程里监听传来的h0

DC_OP_BIND_DRIVER:
  调用driver的bind()

driver's bind op:
  device_add()
    device_add_from_driver()
      devhost_device_add()
        devhost_add()
          创建hrpc, hsend
          给dc发送DC_OP_ADD_DEVICE消息,把hsend发回去,监听hrpc.

coordinator()
首先创建sys, test二个devhost进程。
在遍历驱动的时候,创建root, misc(console)二个devhost 进程

dc prepare proxy引起的devhost的驱动的create方法
第一次调用add device只会把hrpc加入
devhost端的proxy设备里。之后的驱动的create方法中的add device才会
引起dc端的消息。
在dc里,设备都是创建在proxy device下面的。

dmctl可以动态添加driver.

例子:
dc: create device "platform"
dh: add device platform
dc: 匹配到qemu-bus 驱动,让dh那边加载这个驱动
dh: 调用了qemu-bus驱动,里面又有Add device, 回到dc
dc: 创建pci, pl031设备



qemu_bus_bind()
    device_get_protocol(parent)
        dev->ops->get_protocol(dev->ctx == dev)
          platform_bus_get_protocol()
              protocol->ctx= bus = ctx 就是dev
\end{verbatim}

              