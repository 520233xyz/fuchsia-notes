\section{lk\_main}

\subsection{thread\_init\_early}

\verb|thread_construct_first(t, "bootstrap")|, 在主cpu上创建一个反映
当前运行状态的\verb|thread_t|.

\begin{enumerate}
    \item 拿到的cpu num是0
    \item \verb|thread_t|清零
    \item \verb|THREAD_MAGIC|暂时不知道什么用
    \item 设置thread名字
    \item \verb|retcode_wait_queue|的前驱后继都指向自己
    \item 从\verb|tpidr_el1|中拿到\verb|boot_cpu_fake_thread_pointer_location|的地址

    \item \verb|__has_feature(safe_stack)|是clang的特性,对gcc来说,
        \verb|unsafe_sp|就是0(在start.S里安排的)
    \item 把x18保存在\verb|current_percpu_ptr|里
    \item 让\verb|tpidr_el1|指向这个真正的\verb|thread_t|,也就是percpu里的
    \verb|idle_thread|
    \item 把这个thread加到全局的thread list上
\end{enumerate}


\verb|sched_init_early()|初始化各个cpu的run queue为空。

调用C++的全局变量的构造函数\verb|__init_array_start|。

\verb|lk_init_level()|

\verb|required_flag = LK_INIT_FLAG_PRIMARY_CPU|

\verb|start_level = LK_INIT_LEVEL_EARLIEST|

\verb|stop_level = LK_INIT_LEVEL_ARCH_EARLY - 1|

这次没有匹配的init函数可以被调用。(也许我漏掉了?)

\verb|arch_curr_cpu_num_slow()|返回0,因为这时\verb|arm64_cpu_map|还没有初始化。

\verb|arch.cpp: arm64_cpu_early_init()|

把x18指向当前cpu的percpu.

让\verb|VBAR_EL1|指向\verb|exceptions.S|中的\verb|arm64_el1_exception_base|.

把一些cpu feature读到\verb|arm64_features|里

打开DAIF遮掩位。

\verb|main.cpp: platform_early_init()|

\verb|bootdata_paddr|存放的是initrd的加载地址0x48000000。boot-shim会把从device tree
中读出来的initrd的加载地址写入\verb|bootdata_paddr|。此外,qemu会把内核加载到0x40080000处。
内核的实际物理地址就是0x40090000。这些细节需要从qemu的源码中获得。

\verb|boot_reserve_add_range(get_kernel_base_phys(), get_kernel_size())|
把kernel占用的物理地址区域保留起来,也就是0x40090000开始的一块区域。

接下来会处理\verb|append_board_bootdata()|里加入的一些bootdata的section.

然后到\verb|pdev_init()|里调用各个周边设备驱动的初始化代码。

把ramdisk所在的0x48000000开始的一块内存保留起来。

kernel.memory-limit-mb这个命令行参数应该是没有设置的。

在每个物理内存mem arena的尾部放置的是一堆描述page的\verb|vm_page_t|。

把之前标记为保留的内存区域在arena中保留出来,标记成WIRED。

\verb|platform_early_init()|结束。

终于来到了"welcome to Zircon"。

接下来调用\verb|crypto::GlobalPRNG::EarlyBootSeed()|,初始化一些随机数生成器相关的东西。

进入\verb|vm_init_preheap()|,初始化一个唯一的内核虚拟地址空间。

把boot time allocator用掉的内存(主要是分配给translation table)在物理内存管理pmm里标记为已用。

随机保留分配一些页面。可能是为了增加内存随机性。

用dc zva把缓存置零。

进入\verb|heap_init()|,把内核要使用的heap page申请出来。

进入\verb|vm_init()|。为内核占用的虚拟地址区域创建VmObject. VmMapping好像只是记录一个vmo有哪些flags. 在创建VmMapping之后,
Activate()的时候才把它插到parent,也就是RootVmar的ChildList里面。

\verb|platform_init_postvm()|把外围设备的虚拟地址保留起来。

\verb|kernel_init()|, 初始化多处理器的一些状态。初始化每个处理器的timer. 之后,内核就可以创建线程了。缺省的线程stack大小是8k.

线程的切换通过改变x30完成。x30存储的是\verb|arm64_context_switch()|调用的下一条指令的地址。

bootstrap2.

初始化deferred procedure call (dpc)线程。

初始化Handle和Arena. 这个过程没有page committed。内存页真正分配出去是在做Arena::Pool::Pop()的时候。
在VmMappingCoalescer::Flush()里会建立物理页表的映射。在做VmMapping的时候,如果不指定起始地址,
则会去找一个gap地址。

PortDispatcher暂时不知道干啥的。

进入\verb|arch_init()|

给每个cpu创建一个idle thread.  释放副cpu的spin lock.

进入\verb|platform_init()|. 给每个副cpu创建kernel stack, 是一个VmMapping.

\verb|pdev_run_hooks()|

moving to last init leve.

user level init: ktrace, console, userboot.

ktrace: 给它分配一块内存,缺省为32MB

kernel console通过uart读命令。















