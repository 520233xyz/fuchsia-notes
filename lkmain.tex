\section{lkmain}
\begin{verbatim}
thread_init_early()
  thread_construct_first(t, "bootstrap");
    在主cpu上创建一个反映当前运行状态的thread_t
    拿到的cpu num是0
    thread_t清零
    THREAD_MAGIC暂时不知道什么用
    设置thread名字
    retcode_wait_queue的前驱后继都指向自己
    从tpidr_el1中拿到boot_cpu_fake_thread_pointer_location的地址

    __has_feature(safe_stack)是clang的特性,对gcc来说,
        unsafe_sp就是0(在start.S里安排的)
    把x18保存在current_percpu_ptr里
    让tpidr_el1指向这个真正的thread_t,也就是percpu里的
    idle_thread
    把这个thread加到全局的thread list上

sched_init_early()初始化各个cpu的run queue为空。

调用C++的全局变量的构造函数__init_array_start。

lk_init_level()
  required_flag = LK_INIT_FLAG_PRIMARY_CPU
  start_level = LK_INIT_LEVEL_EARLIEST
  stop_level = LK_INIT_LEVEL_ARCH_EARLY - 1
  这次没有匹配的init函数可以被调用。(也许我漏掉了?)

arch_early_init();
x64:
  x86_mmu_early_init();
    x86_mmu_percpu_init();
      设置cr0, cr4
arm64:
  arch_curr_cpu_num_slow()返回0,因为这时arm64_cpu_map还没有初始化。
  arch.cpp: arm64_cpu_early_init()
  把x18指向当前cpu的percpu.
  让VBAR_EL1指向exceptions.S中的arm64_el1_exception_base.
  把一些cpu feature读到arm64_features里
  打开DAIF遮掩位。

lk_primary_cpu_init_level(LK_INIT_LEVEL_ARCH_EARLY, LK_INIT_LEVEL_PLATFORM_EARLY - 1);

platform_early_init()
x64:
  pc_init_debug_default_early();
  platform_save_bootloader_data();
    multiboot_info_t* mi = (multiboot_info_t*)X86_PHYS_TO_VIRT(_multiboot_info);
    process_zbi(reinterpret_cast<zbi_header_t*>(X86_PHYS_TO_VIRT(mod->mod_start)),mod->mod_start);
      process_zbi_item()
        bootloader会把efi mem map信息放到ramdisk的前面: bootloader/src/zircon.c

  pc_init_debug_early();
    handle_serial_cmdline()
      bootloader.uart.type = ZBI_UART_PC_PORT;
      bootloader.uart.base = 0x3f8;
      bootloader.uart.irq = ISA_IRQ_SERIAL1;
    uart_io_port = static_cast<uint32_t>(bootloader.uart.base);
    uart_irq = bootloader.uart.irq;
  platform_early_display_init();
  boot_reserve_init();
    boot_reserve_add_range(get_kernel_base_phys(), get_kernel_size());
  platform_preserve_ramdisk();

  pc_mem_init();
    platform_mem_range_init()
      efi_range_init(&range, &efi_seq);
        mem_arena_init(&range);
          这里通过UEFI提供的信息初始化物理内存区域。具体参见UEFI标准和bootloader传递memory map
          信息的代码。

        
    efi_range_init(&range, &efi_seq)
    e820_range_init(&range, &e820_seq)
    multiboot_range_init(&range, &multiboot_seq)
      这里会把内存区域信息存一份。后面x86_pcie_init_hook()会用到

  boot_reserve_wire();
    pmm_alloc_range(res[i].pa, pages, &reserved_page_list);
      // walk through the arenas, looking to see if the physical page belongs to it
      把之前保留的物理页标记成wired

lk_primary_cpu_init_level(LK_INIT_LEVEL_PLATFORM_EARLY, LK_INIT_LEVEL_TARGET_EARLY - 1);
  counters_init()

target_early_init();
  none

lk_primary_cpu_init_level(LK_INIT_LEVEL_TARGET_EARLY, LK_INIT_LEVEL_VM_PREHEAP - 1);
  crypto::GlobalPRNG::EarlyBootSeed

vm_init_preheap();
  VmAspace::KernelAspaceInitPreHeap();
    static VmAspace _kernel_aspace;
    static VmAddressRegionDummy dummy_vmar;
    static VmAddressRegion _kernel_root_vmar(_kernel_aspace);
    _kernel_aspace.root_vmar_ = fbl::AdoptRef(&_kernel_root_vmar);
    _kernel_aspace.Init();
      arch_aspace_.Init(base_, size_, arch_aspace_flags);
      X86ArchVmAspace::Init(vaddr_t base, size_t size, uint mmu_flags)
        mmu = new (page_table_storage_) X86PageTableMmu();
        mmu->InitKernel(this);
          ctx_ = ctx;   X86ArchVmAspace
          pages_ = 1;
          使用start.S里设置的页表

  // mark the physical pages used by the boot time allocator

  // grab a page and mark it as the zero page

lk_primary_cpu_init_level(LK_INIT_LEVEL_VM_PREHEAP, LK_INIT_LEVEL_HEAP - 1);
  none

heap_init();
  cmpct_init();
    heap_grow(initial_alloc);
      heap_page_alloc(size >> PAGE_SIZE_SHIFT);
        pmm_alloc_contiguous(pages, 0, PAGE_SIZE_SHIFT, &pa, &list);
          分配物理页给内核堆
          问题:如何保证分配的物理页不和之后的设备的mmio的地址冲突?

lk_primary_cpu_init_level(LK_INIT_LEVEL_HEAP, LK_INIT_LEVEL_VM - 1);
  console_init()
    init_history();
      history = calloc(1, HISTORY_LEN * LINE_LEN);
      history_next = 0;

vm_init();
  ProtectRegion(aspace, region->base, region->arch_mmu_flags);
    vm_mapping->Protect(vm_mapping->base(), vm_mapping->size(), arch_mmu_flags);
      ProtectLocked(base, size, new_arch_mmu_flags);
        ProtectOrUnmap(aspace_, base, size, new_arch_mmu_flags);
          aspace->arch_aspace().Protect(base, size / PAGE_SIZE, new_arch_mmu_flags);
            pt_->ProtectPages(vaddr, count, mmu_flags);
              UpdateMapping(virt_, mmu_flags, top_level(), start, &result, &cm);
  aspace->ReserveSpace("physmap", PHYSMAP_SIZE, PHYSMAP_BASE);
    保留一部分虚拟地址区域,对应于一些有用的物理地址。x86是64gb, arm64是512gb
    // The kernel physmap is a region of the kernel where all of useful physical memory
    // is mapped in one large chunk. It's up to the individual architecture to decide
    // how much to map but it's usually a fairly large chunk at the base of the kernel
    // address space.
    目的是一些需要用到物理地址的情况,比如设备内存映射,可以直接在虚拟地址和物理地址之间建立关系。
    kernel heap也是利用这个映射区域。
    内核自己要用的其他虚拟内存,比如后面的Handle arena,都在这个区域之后映射。
    之所以Handle arena不用physmap, 是因为它用了vmo lazy commit, 用了vmo就无法用physmap.
    而kernel heap不需要用vmo来back up它。

    trim_to_aspace(*this, vaddr, size);
      检查溢出的情况
    VmObjectPaged::Create(PMM_ALLOC_FLAG_ANY, 0, &vmo);  
      创建一个长度为0的vmo

    MapObjectInternal(fbl::move(vmo), name, 0, size, &ptr, 0, VMM_FLAG_VALLOC_SPECIFIC,
      arch_mmu_flags);
      offset = 0, size = 64gb, ptr = KERNEL_BASE

      vmar_offset = reinterpret_cast<vaddr_t>(*ptr);
      vmar_offset -= base_; // 0
      RootVmar()->CreateVmMapping(vmar_offset, size, align_pow2,
                                                     vmar_flags,
                                                     vmo, offset, arch_mmu_flags, name, &r);
      CreateSubVmarInternal(mapping_offset, size, align_pow2, vmar_flags, fbl::move(vmo),
                                                     vmo_offset, arch_mmu_flags, name, &res);
      vmar = fbl::AdoptRef(new (&ac) VmMapping(*this, new_base, size, vmar_flags,
                                                fbl::move(vmo), vmo_offset, arch_mmu_flags));
      
      vmar->Activate();
        parent_->subregions_.insert(fbl::RefPtr<VmAddressRegionOrMapping>(this));
      
      
      no commit.

lk_primary_cpu_init_level(LK_INIT_LEVEL_VM, LK_INIT_LEVEL_KERNEL - 1)
  pmm_enforce_fill(uint level)
    pmm_node.EnforceFill();
      用0x42填充内存

  platform_ensure_display_memtype


  //initialize ACPI tables as soon as we have a working VM
  platform_init_acpi_tables(uint level)
    This function enables *only* the ACPICA Table Manager subsystem.
    The rest of the ACPI subsystem will remain uninitialized.

    AcpiInitializeTables(acpi_tables, ACPI_MAX_INIT_TABLES, FALSE);
      acpi在拿到rsdp的物理地址后,会把它映射到一个新的虚拟地址上。
      acpi table manager可以独立初始化
  
  //Begin running after ACPI tables are up
  hpet_init(uint level)
    为hept映射一个页的地址

  platform_init_apic(uint level)
    通过设置LVT可以指定各种中断的向量号
    把8259a关掉

    platform_enumerate_io_apics(NULL, 0, &num_io_apics);
      acpi_get_madt_record_limits(&records_start, &records_end);
      获得ioapic的信息

    apic_vm_init();
      // Create a mapping for the page of MMIO registers
    apic_local_init();
      x2apic_enabled = true;
      // Enter xAPIC or x2APIC mode and set the base address

    apic_io_init(io_apics, num_io_apics, isos, num_isos);
      为每个ioapic映射物理地址

      global irq是系统给硬件配置的中断号,acpi会配置这个。
      io apic会把global irq映射到x86 vector,一个新的中断号。这个中断号是cpu
      看到的,调用哪个中断handler是根据这个中断号来的。

      register_int_handler(unsigned int vector, int_handler handler, void* arg)
      这个函数的逻辑是:vector是global irq, 先检查其对应的x86 vector是否合法,
      如果不合法则分配一个新的x86 vector, 然后把handler配置到x86 vector上,
      最后在io apic里配置global irq到x86 vector的重定向。

  pc_init_timer(uint level)

kernel_init();
  mp_init();
  thread_init();
  timer_queue_init();
  // nothing interesting

lk_primary_cpu_init_level(LK_INIT_LEVEL_KERNEL, LK_INIT_LEVEL_THREADING - 1);
  dlog_init_hook
  thread_set_priority_experiment_init_hook
  crypto::GlobalPRNG::BecomeThreadSafe

bootstrap2 thread:

lk_primary_cpu_init_level(LK_INIT_LEVEL_THREADING, LK_INIT_LEVEL_ARCH - 1);

  dpc_init

  object_glue_init(uint level)
    Handle::Init();
      arena_.Init("handles", sizeof(Handle), kMaxHandleCount);
        VmObjectPaged::Create(PMM_ALLOC_FLAG_ANY, vmo_sz, &vmo);
          root_vmar->CreateSubVmar(0, // offset (ignored)
              vmar_sz,
              false, // align_pow2
              VMAR_FLAG_CAN_MAP_READ |
                  VMAR_FLAG_CAN_MAP_WRITE |
                  VMAR_FLAG_CAN_MAP_SPECIFIC,
              vname, &vmar);

              AllocSpotLocked(size, align_pow2, arch_mmu_flags, &new_base);
                LinearRegionAllocatorLocked(size, align_pow2, arch_mmu_flags, spot);
                  CheckGapLocked(before_iter, after_iter, spot, base, align, size, 0, arch_mmu_flags)
    arena映射的是之前kernel root_vmar保留区域之后的区域。
          vmar->CreateVmMapping(0, // mapping_offset
                               control_mem_sz,
                               false, // align_pow2
                               VMAR_FLAG_SPECIFIC,
                               vmo,
                               0, // vmo_offset
                               ARCH_MMU_FLAG_PERM_READ |
                                   ARCH_MMU_FLAG_PERM_WRITE,
                               "control", &control_mapping);

arch_init();
  x86_feature_debug();

  x86_mmu_init();

  idt_setup_readonly();

  x86_perfmon_init();
  x86_processor_trace_init();

  // nothing interesting

lk_primary_cpu_init_level(LK_INIT_LEVEL_ARCH, LK_INIT_LEVEL_PLATFORM - 1);
  // none

platform_init();

  platform_init_smp();
    x86_init_smp(apic_ids.get(), num_cpus);
      lk_init_secondary_cpus(num_cpus - 1);

    x86_bringup_aps(apic_ids.get(), num_cpus - 1);
      x86_bootstrap16_acquire((uintptr_t)_x86_secondary_cpu_long_mode_entry,
                                     &bootstrap_aspace, (void**)&bootstrap_data,
                                     &bootstrap_instr_ptr);
      apic_send_ipi(vec, apic_id, DELIVERY_MODE_STARTUP);
      // The effect is to set CS:IP to VV00:0000h
      // AP 实模式的启动代码是之前扫描内存时选择的低于1MB的一个地方,启动数据在4kb之后。
      // far return的时候,%esp指向bootstrap data里面的phys_long_mode_entry



lk_primary_cpu_init_level(LK_INIT_LEVEL_PLATFORM, LK_INIT_LEVEL_TARGET - 1);
  x86_pcie_init_hook()
    PcieBusDriver::InitializeDriver(platform_pcie_support);

      driver_ = fbl::AdoptRef(new (&ac) PcieBusDriver(platform));
      driver_->AllocBookkeeping();

target_init();
lk_primary_cpu_init_level(LK_INIT_LEVEL_TARGET, LK_INIT_LEVEL_LAST);

-----------
以下是老内容。

bootdata_paddr存放的是initrd的加载地址0x48000000。boot-shim会把从device tree
中读出来的initrd的加载地址写入bootdata_paddr。此外,qemu会把内核加载到0x40080000处。
内核的实际物理地址就是0x40090000。这些细节需要从qemu的源码中获得。

boot_reserve_add_range(get_kernel_base_phys(), get_kernel_size())
把kernel占用的物理地址区域保留起来,也就是0x40090000开始的一块区域。

接下来会处理append_board_bootdata()里加入的一些bootdata的section.

然后到pdev_init()里调用各个周边设备驱动的初始化代码。

把ramdisk所在的0x48000000开始的一块内存保留起来。

kernel.memory-limit-mb这个命令行参数应该是没有设置的。

在每个物理内存mem arena的尾部放置的是一堆描述page的vm_page_t。

把之前标记为保留的内存区域在arena中保留出来,标记成WIRED。

platform_early_init()结束。

终于来到了"welcome to Zircon"。

接下来调用crypto::GlobalPRNG::EarlyBootSeed(),初始化一些随机数生成器相关的东西。

进入vm_init_preheap(),初始化一个唯一的内核虚拟地址空间。

把boot time allocator用掉的内存(主要是分配给translation table)在物理内存管理pmm里标记为已用。

随机保留分配一些页面。可能是为了增加内存随机性。

用dc zva把缓存置零。

进入heap_init(),把内核要使用的heap page申请出来。

进入vm_init()。为内核占用的虚拟地址区域创建VmObject. VmMapping好像只是记录一个vmo有哪些flags. 
在创建VmMapping之后,
Activate()的时候才把它插到parent,也就是RootVmar的ChildList里面。

platform_init_postvm()把外围设备的虚拟地址保留起来。

kernel_init(), 初始化多处理器的一些状态。初始化每个处理器的timer. 之后,内核就可以创建线程了。缺省的线程stack大小是8k.

线程执行轨迹的切换通过改变x30完成。x30是link register,
存储的是arm64_context_switch()函数调用语句的下一条指令的地址。
当在arm64_context_switch()里面切换了sp之后,新的x30从新线程的栈上pop出来。
ret指令接着x30存储的地址继续执行。实际上,每个线程的x30存储的地址都是一样的,因为
arm64_context_switch()只在一个地方被调用。

bootstrap2.

初始化deferred procedure call (dpc)线程。

初始化Handle和Arena. 这个过程没有page committed。内存页真正分配出去是在做Arena::Pool::Pop()的时候。
在VmMappingCoalescer::Flush()里会建立物理页表的映射。在做VmMapping的时候,如果不指定起始地址,
则会去找一个gap地址。

PortDispatcher暂时不知道干啥的。

进入arch_init()

给每个cpu创建一个idle thread.  释放副cpu的spin lock.

进入platform_init(). 给每个副cpu创建kernel stack, 是一个VmMapping.

pdev_run_hooks()

moving to last init leve.

user level init: ktrace, console, userboot.

ktrace: 给它分配一块内存,缺省为32MB

kernel console通过uart读命令。

\end{verbatim}













