\section{虚拟内存管理}
内核的页表都是在start.S里映射好的。

在vm\_init()里会保留起来。

kernel heap直接从物理内存拿页。因为内核页表是都映射好的,所以不用再次映射。

内核另外一个用内存的地方是handle arena. 那个地方会申请物理页,然后commit.


zircon的每个进程的页表对应的page是初始化process handle时创建的。页表地址
会记录在aspace里面。

当切换进程后,如果通过channel传递过来的vmar里面已经映射的vmo的地址没有对应的物理页面,
会触发Page fault。pagefault会优先从vmo里拿它已有的页面,然后到页表里建立映射关系。

当load devmgr的时候,如果有动态链接器,则entry会设置为interp的入口。

gold对于aarch64默认entry symbol是\verb|_start|

vmobject可以从一个地址和size创建出来。也可以clone出来。

ld.so会把application的phdr全部映射一遍,其中有一个segment包含ehdr, 从中得到entry addr (需要用映射到的地址进行修正)



在进程创建的时候,会给它一个root VMAR handle.