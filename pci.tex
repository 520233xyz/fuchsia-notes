\section{pci}

pci firmware specification rev 3.0 (pdf)

The base address returned by MCFG table for a given PCI Segment group is always
with respect to bus 0 as specified in the PCI Express Base Specification (and
the PCI-X Specification). It is the responsibility of system software to
calculate the start and end of the supported memory mapped configuration address
range based on the start and end bus numbers specified in the MCFG entry. System
software must make no assumptions about the memory range corresponding to the
base address up to the start of the memory mapped configuration space (as
specified by start bus number).

系统会给pci的每个bus的每个device的每个function分配一个4k配置空间。最多是256MB的
连续配置空间。这个配置空间的基地址由acpi提供。

在fuchsia里只有一个pci root设备,不管有多少个bus.

acpi首先通过系统调用初始化内核里的pci驱动。然后再添加devmgr里的pci驱动。

每个pci设备都有一个独立的proxy进程,以及内核里的一个PciDeviceDispatcher

\begin{verbatim}
acpi_drv_create()
  get_pci_init_arg(&arg, &arg_size);
    find_pcie_config(res);
    find_pci_legacy_irq_mapping(res);

  sys_pci_init()
    首先处理legacy irq

    bus_start == 0
    pcie->AddEcamRegion(ecam);

    PcieRootLUTSwizzle::Create(*pcie, 0, arg->dev_pin_to_global_irq);
    pcie->AddRoot(fbl::move(root));
    pcie->StartBusDriver();
      root->ScanDownstream();
        PcieUpstreamNode::ScanDownstream()
          cfg = driver().GetConfig(managed_bus_id_, dev_id, func_id);


kpci.c:
pci_drv_bind(void* ctx, zx_device_t* parent)
  ctx == null
  pci_init_child(parent, index)
    zx_pci_get_nth_device(get_root_resource(), index, &info, &handle);
    sys_pci_get_nth_device(zx_handle_t hrsrc,
                                   uint32_t index,
                                   user_out_ptr<zx_pcie_device_info_t> out_info,
                                   user_out_handle* out_handle)
      PciDeviceDispatcher::Create(index, &info, &dispatcher, &rights);
        bus_drv->GetNthDevice(index);
          PcieBusDriver::ForeachDevice(ForeachDeviceCallback cbk, void* ctx)
            ForeachRoot(ForeachRootCallback cbk, void* ctx)


\end{verbatim}