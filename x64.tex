\section{x64 boot}

.global makes the symbol visible to ld. If you define symbol in your partial
program, its value is made available to other partial programs that are linked
with it. Otherwise, symbol takes its attributes from a symbol of the same name
from another file linked into the same program. 

The directive sets the visibility to hidden which means that the symbols are
not visible to other components.

--build-id Request the creation of a .note.gnu.build-id ELF note section or a
.buildid COFF section. The contents of the note are unique bits identifying this
linked file.

与arm64相比,x64 qemu没有shim,直接用的zircon.bin作为kernel.

对于.o文件,一个symbol是符号表的一项。它的value是一个偏移量。最终到了exe文件里,这个value
会被替换成最终的地址。

\begin{verbatim}
qemu x86会在内核文件的前8192字节遍历寻找MULTIBOOT MAGIC.  
IMAGE_ELF_ENTRY 实际上没用。

IMAGE_MULTIBOOT_ENTRY = _multiboot_start

x64内核真正的入口是_multiboot_start


KERNEL_BASE := 0xffffffff00000000
__code_start = KERNEL_BASE
IMAGE_LOAD_START = __code_start
IMAGE_ELF_ENTRY = _start

qemu-system-x86_64 -m 2048 -nographic -net none -smp 4,threads=2 
  -machine q35 -kernel /home/xzx/zircon/build-x64/zircon.bin 
  -device isa-debug-exit,iobase=0xf4,iosize=0x04 
  -cpu Haswell,+smap,-check,-fsgsbase 
  -initrd /home/xzx/zircon/build-x64/bootdata.bin 
  -append 'kernel.serial=legacy TERM=xterm-256color 
  kernel.entropy-mixin=9b158397dbade73e8784d4e783af93847ebb1d1be402fa73d53b0e2113e68200 
  kernel.halt-on-panic=true '

zircon.bin:

x86_64-elf-objcopy -O binary build-x64/zircon-image.elf build-x64/zircon.bin

zircon-image.elf:
x86_64-elf-ld -nostdlib --build-id 
  -z noexecstack 
  -z max-page-size=4096 
  --gc-sections 
  --build-id=none \
  -o build-x64/zircon-image.elf 
  -T kernel/image.ld 
  --just-symbols ./build-x64/zircon.elf
  ./build-x64/kernel-vars.ld ./build-x64/zircon.image.o

使用kernel/image.ld脚本构建最后的内核image

zircon.image.o:
LKNAME := zircon
OUTLKELF := $(BUILDDIR)/$(LKNAME).elf
OUTLKELF_RAW := $(OUTLKELF).bin
KERNEL_IMAGE='"$(OUTLKELF_RAW)"'

x86_64-elf-gcc   \
    -O2 -g -fdebug-prefix-map=/home/xzx/zircon=. 
    -finline 
    -include ./build-x64/config-global.h 
    -Wall -Wextra -Wno-multichar -Werror -Wno-error=deprecated-declarations 
    -Wno-unused-parameter -Wno-unused-function -Werror=unused-label 
    -Werror=return-type -fno-common -ffunction-sections 
    -fdata-sections -Wno-nonnull-compare -march=x86-64 
    -mcx16 -malign-data=abi -ffreestanding 
    -include ./build-x64/config-kernel.h 
    -Wformat=2 -Wvla -Wformat-signedness 
    -fno-exceptions -fno-unwind-tables 
    -fno-omit-frame-pointer -falign-jumps=1 
    -falign-loops=1 -falign-functions=4 
    -msoft-float -mno-mmx -mno-sse -mno-sse2 
    -mno-3dnow -mno-avx -mno-avx2 -mno-80387 
    -mno-fp-ret-in-387 -fPIE -include kernel/include/hidden.h 
    -mno-red-zone -mskip-rax-setup 
    -c kernel/arch/x86/image.S 
    -MD -MP -MT build-x64/zircon.image.o -MF build-x64/zircon.image.d 
    -o build-x64/zircon.image.o


zircon.elf.bin:
x86_64-elf-objcopy -O binary build-x64/zircon.elf build-x64/zircon.elf.bin


zircon.elf:
./prebuilt/downloads/gcc/bin/x86_64-elf-ld 
  -nostdlib 
  --build-id 
  -z noexecstack 
  -z max-page-size=4096 
  --gc-sections  
  --emit-relocs 
  -T kernel/kernel.ld 
  build-x64/kernel-vars.ld 
  build-x64/kernel/target/pc/pc.mod.o 
  build-x64/kernel/platform/pc/pc.mod.o 
  build-x64/kernel/arch/x86/x86.mod.o 
  build-x64/kernel/top/top.mod.o 
  build-x64/kernel/arch/x86/page_tables/page_tables.mod.o 
  build-x64/kernel/dev/intel_rng/intel_rng.mod.o 
  build-x64/kernel/dev/interrupt/interrupt.mod.o 
  build-x64/kernel/dev/iommu/dummy/dummy.mod.o 
  build-x64/kernel/dev/iommu/intel/intel.mod.o 
  build-x64/kernel/dev/pcie/pcie.mod.o 
  build-x64/kernel/kernel/kernel.mod.o 
  build-x64/kernel/lib/bitmap/bitmap.mod.o 
  build-x64/kernel/lib/cbuf/cbuf.mod.o 
  build-x64/kernel/lib/code_patching/code_patching.mod.o 
  build-x64/kernel/lib/debugcommands/debugcommands.mod.o 
  build-x64/kernel/lib/debuglog/debuglog.mod.o 
  build-x64/kernel/lib/fbl/fbl.mod.o 
  build-x64/kernel/lib/fixed_point/fixed_point.mod.o 
  build-x64/kernel/lib/gfxconsole/gfxconsole.mod.o 
  build-x64/kernel/lib/ktrace/ktrace.mod.o 
  build-x64/kernel/lib/memory_limit/memory_limit.mod.o 
  build-x64/kernel/lib/mtrace/mtrace.mod.o 
  build-x64/kernel/lib/pow2_range_allocator/pow2_range_allocator.mod.o 
  build-x64/kernel/lib/smbios/smbios.mod.o 
  build-x64/kernel/lib/userboot/userboot.mod.o 
  build-x64/kernel/lib/version/version.mod.o 
  build-x64/kernel/lib/zbi/zbi.mod.o 
  build-x64/kernel/object/object.mod.o 
  build-x64/kernel/platform/platform.mod.o 
  build-x64/kernel/syscalls/syscalls.mod.o 
  build-x64/kernel/target/target.mod.o 
  build-x64/kernel/tests/tests.mod.o 
  build-x64/third_party/lib/acpica/acpica.mod.o 
  build-x64/third_party/lib/cksum/cksum.mod.o 
  build-x64/kernel/dev/hw_rng/hw_rng.mod.o 
  build-x64/kernel/dev/udisplay/udisplay.mod.o 
  build-x64/kernel/lib/console/console.mod.o 
  build-x64/kernel/lib/counters/counters.mod.o 
  build-x64/kernel/lib/crypto/crypto.mod.o 
  build-x64/kernel/lib/debug/debug.mod.o 
  build-x64/kernel/lib/explicit-memory/explicit-memory.mod.o 
  build-x64/kernel/lib/gfx/gfx.mod.o 
  build-x64/kernel/lib/header_tests/header_tests.mod.o 
  build-x64/kernel/lib/heap/heap.mod.o 
  build-x64/kernel/lib/hwreg/hwreg.mod.o 
  build-x64/kernel/lib/hypervisor/hypervisor.mod.o 
  build-x64/kernel/lib/libc/libc.mod.o 
  build-x64/kernel/lib/oom/oom.mod.o 
  build-x64/kernel/lib/pci/pci.mod.o 
  build-x64/kernel/lib/region-alloc/region-alloc.mod.o 
  build-x64/kernel/lib/unittest/unittest.mod.o 
  build-x64/kernel/lib/user_copy/user_copy.mod.o 
  build-x64/kernel/lib/vdso/vdso.mod.o 
  build-x64/kernel/lib/zxcpp/zxcpp.mod.o 
  build-x64/kernel/vm/vm.mod.o 
  build-x64/kernel/arch/x86/hypervisor/hypervisor.mod.o 
  build-x64/kernel/lib/heap/cmpctmalloc/cmpctmalloc.mod.o 
  build-x64/kernel/lib/io/io.mod.o 
  build-x64/kernel/lib/pretty/pretty.mod.o 
  build-x64/third_party/lib/cryptolib/cryptolib.mod.o 
  build-x64/third_party/lib/jitterentropy/jitterentropy.mod.o 
  build-x64/third_party/lib/qrcodegen/qrcodegen.mod.o 
  build-x64/third_party/lib/uboringssl/uboringssl.mod.o 
  -o build-x64/zircon.elf

*.o -> zircon.elf -> zircon.elf.bin + image.S -> zircon.image.o -> zircon-image.elf -> zircon.bin


echo "INPUT(build-x64/kernel/arch/x86/kernel/arch/x86/bp_percpu.c.o 
build-x64/kernel/arch/x86/kernel/arch/x86/arch.cpp.o 
build-x64/kernel/arch/x86/kernel/arch/x86/cache.cpp.o 
build-x64/kernel/arch/x86/kernel/arch/x86/cpu_topology.cpp.o 
build-x64/kernel/arch/x86/kernel/arch/x86/debugger.cpp.o 
build-x64/kernel/arch/x86/kernel/arch/x86/descriptor.cpp.o 
build-x64/kernel/arch/x86/kernel/arch/x86/faults.cpp.o 
build-x64/kernel/arch/x86/kernel/arch/x86/feature.cpp.o 
build-x64/kernel/arch/x86/kernel/arch/x86/hwp.cpp.o 
build-x64/kernel/arch/x86/kernel/arch/x86/idt.cpp.o
build-x64/kernel/arch/x86/kernel/arch/x86/ioapic.cpp.o 
build-x64/kernel/arch/x86/kernel/arch/x86/ioport.cpp.o 
build-x64/kernel/arch/x86/kernel/arch/x86/lapic.cpp.o 
build-x64/kernel/arch/x86/kernel/arch/x86/mmu.cpp.o 
build-x64/kernel/arch/x86/kernel/arch/x86/mmu_mem_types.cpp.o 
build-x64/kernel/arch/x86/kernel/arch/x86/mmu_tests.cpp.o 
build-x64/kernel/arch/x86/kernel/arch/x86/mp.cpp.o 
build-x64/kernel/arch/x86/kernel/arch/x86/perf_mon.cpp.o 
build-x64/kernel/arch/x86/kernel/arch/x86/proc_trace.cpp.o 
build-x64/kernel/arch/x86/kernel/arch/x86/pvclock.cpp.o 
build-x64/kernel/arch/x86/kernel/arch/x86/registers.cpp.o 
build-x64/kernel/arch/x86/kernel/arch/x86/thread.cpp.o 
build-x64/kernel/arch/x86/kernel/arch/x86/timer_freq.cpp.o
build-x64/kernel/arch/x86/kernel/arch/x86/tsc.cpp.o 
build-x64/kernel/arch/x86/kernel/arch/x86/user_copy.cpp.o 
build-x64/kernel/arch/x86/kernel/arch/x86/bootstrap16.cpp.o 
build-x64/kernel/arch/x86/kernel/arch/x86/smp.cpp.o 
build-x64/kernel/arch/x86/kernel/arch/x86/acpi.S.o 
build-x64/kernel/arch/x86/kernel/arch/x86/asm.S.o 
build-x64/kernel/arch/x86/kernel/arch/x86/exceptions.S.o 
build-x64/kernel/arch/x86/kernel/arch/x86/gdt.S.o 
build-x64/kernel/arch/x86/kernel/arch/x86/mexec.S.o 
build-x64/kernel/arch/x86/kernel/arch/x86/ops.S.o 
build-x64/kernel/arch/x86/kernel/arch/x86/start.S.o 
build-x64/kernel/arch/x86/kernel/arch/x86/syscall.S.o 
build-x64/kernel/arch/x86/kernel/arch/x86/user_copy.S.o 
build-x64/kernel/arch/x86/kernel/arch/x86/uspace_entry.S.o 
build-x64/kernel/arch/x86/kernel/arch/x86/start16.S.o)" 
> build-x64/kernel/arch/x86/x86.mod.o


\end{verbatim}

\section{Intel 64架构}

real mode => protected mode => IA-32e mode
interrupt是硬件中断。exception是软件中断。

idt: task, interrupt, trap


\section{启动过程}


\begin{verbatim}
在qemu multiboot.S里
movl		$MULTIBOOT_MAGIC, %eax

在start.S里,
cmpl $MULTIBOOT_BOOTLOADER_MAGIC, %eax

load_phys_gdtr:
加载GDT
DATA_SELECTOR 作为各种selector

mov $PHYS(_kstack_end), %esp  设置stack pointer

打开paging
Preparing 64 bit paging, we will use 2MB pages covering 1GB
for initial bootstrap, this page table will be 1 to 1

_end是整个内核映像结束的位置。在kernel.ld里定义。

mov $X86_MSR_IA32_GS_BASE, %ecx
wrmsr
把bp_percpu的地址写入%gs.base

加载IDT

x86_init_percpu
  当前的percpu指向percpu结构. kernel gs msr清零。
  之后swapgs会设置gs寄存器的值为kernel gs msr的值

  x86_feature_init();
    获得cpuid的各种信息

  x86_cpu_topology_init()

  x86_extended_register_init();
    x86_extended_register_cpu_init();
      x86_set_cr4(cr4 | X86_CR4_OSXSAVE);
  x86_extended_register_enable_feature(X86_EXTENDED_REGISTER_SSE);
  x86_extended_register_enable_feature(X86_EXTENDED_REGISTER_AVX);

  // This can be turned on/off later by the user. Turn it on here so that
  // the buffer size assumes it's on.
  x86_extended_register_enable_feature(X86_EXTENDED_REGISTER_PT);
  // But then set the default mode to off.
  x86_set_extended_register_pt_state(false);

  x86_initialize_percpu_tss();
    每个cpu有一个缺省的tss
    x86_tss_assign_ists(percpu, tss);
    tss里目前看只用了ist
    在遗产模式下,可以用task gate来获得中断执行的好的stack。但是64位模式不行,所以在tss里
    提供多个ist,供不同的中断使用。

  // Setup the post early boot IDT
  idt_setup(&_idt);


  设置syscall地址。



\end{verbatim}

\section{x64 中断处理}
\begin{verbatim}
x86_exception_handler(x86_iframe_t* frame)
  int_handler_start(&state);
    arch_set_in_int_handler(true);
      WRITE_PERCPU_FIELD32(in_irq, value);
      x86_write_gs_offset32(offsetof(struct x86_percpu, field), (value))
         __asm__( "movl   %0, %%gs:%1" : : "ir" (val), "m" (*(uint32_t *)(offset)) : "memory");

  is_from_user(frame);
    SELECTOR_PL(frame->cs) != 0;

  handle_exception_types(frame);

  do_preempt = int_handler_finish(&state);



\end{verbatim}