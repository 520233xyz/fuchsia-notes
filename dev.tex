\section{Development}

zedboot means: after booting the target, the target will through network get the real 
OS files on to the internal storage.

问题1: 没有特殊的bootloader能否启动vim2

问题2:zbi是否会根据-o 里面的board name来选择输入文件。不会。vim2不需要特殊的bootimage.

问题3: fastboot flash 后面的参数boot part有哪几种。是什么意思。emmc如何分区的。在安卓里,
  fastboot flash recovery xxx.img
在flash-hikey里,有很多种。


fastboot是通过usb-c线flash过去的。

串口应该只是输出调试信息

zedboot是通过网络传输文件的。

板子上的emmc有一些分区。这些分区也可以用sd card启动之后,用uboot重新分。

fastboot是一种协议,可以进入fastboot模式,之后可以flash各种分区。包括bootloader和boot. 
分区用名字来指定。

fastboot模式是bootloader的一种模式,有多种方式可以进入。要素就是要在bootloader(比如uboot)的启动
过程中,给它一个信号,比如按住某个键,短接某个跳线,或者是从串口给它一个信号,它就会进入fastboot模式。
如果没有信号,它就正常启动,加载内核。

先更新bootloader,再更新boot.img. 然后启动到zedboot模式下面,主机要启动好fx boot服务器。

启动到zedboot模式之后,会自动开始pave. 之后完整的系统就安装到emmc上面了。

\section{板子}

first, the SoC loads the BL1 (first lever bootloader) from maskrom, 
then depending on state described by the flowchart in the S912 datasheet, 
BL1 loads and authenticates BL2 from external storage.

BL2 is mainly in charge of setting up DDR and clocks, then it loads and 
authenticates BL30 from the same storage as itself.

BL30 is in charge of various things like setting up (undisclosed) security 
stuff and loading the firmware of the cortex-M3 core that handles power-management and CPU cores DVFS (also known as SCPI in ARM terminology).

BL2 also loads/authenticates BL31, also called the secure monitor, running 
in trustzone and in charge of providing secure services to the main OS, and 
finally BL33, which is u-boot.

通过串口可以看到启动信息:


bl30, bl31 and bl33 are stored in a FIP image on eMMC.
you can see that along with the flash offsets in the serial 
console traces output by BL2 just before u-boot starts.

eg. on the kvim2 pro:
\begin{verbatim}
Load fip header from eMMC, src: 0x0000c200, des: 0x01400000, size: 0x00004000
New fip structure!
Load bl30 from eMMC, src: 0x00010200, des: 0x01100000, size: 0x0000d600
Load bl31 from eMMC, src: 0x00020200, des: 0x05100000, size: 0x0002c600
Load bl33 from eMMC, src: 0x00050200, des: 0x01000000, size: 0x00055000
\end{verbatim}
the ‘src’ values are offset in eMMC (probably bytes, because all
values are aligned to 0x200, which is standard sector size), and
‘des’ are RAM addresses.

文档地址:
\verb|https://github.com/ARM-software/arm-trusted-firmware/tree/master/docs|

通过adb shell或者u-boot的命令行可以看到emmc的分区。

\begin{verbatim}
dmesg | grep mmcblk0p

[ 2.849364@2] [mmcblk0p01] bootloader offset 0x000000000000, size 0x000000400000
[ 2.849511@1] [mmcblk0p02] reserved offset 0x000002400000, size 0x000004000000
[ 2.849645@2] [mmcblk0p03] cache offset 0x000006c00000, size 0x000020000000
[ 2.849773@1] [mmcblk0p04] env offset 0x000027400000, size 0x000000800000
[ 2.849897@2] [mmcblk0p05] logo offset 0x000028400000, size 0x000002000000
[ 2.850021@1] [mmcblk0p06] recovery offset 0x00002ac00000, size 0x000002000000
[ 2.850156@2] [mmcblk0p07] rsv offset 0x00002d400000, size 0x000000800000
[ 2.850290@1] [mmcblk0p08] tee offset 0x00002e400000, size 0x000000800000
[ 2.850422@2] [mmcblk0p09] crypt offset 0x00002f400000, size 0x000002000000
[ 2.850559@1] [mmcblk0p10] misc offset 0x000031c00000, size 0x000002000000
[ 2.850691@2] [mmcblk0p11] boot offset 0x000034400000, size 0x000002000000
[ 2.850830@1] [mmcblk0p12] system offset 0x000036c00000, size 0x000080000000
[ 2.850959@2] [mmcblk0p13] data offset 0x0000b7400000, size 0x000dd8400000

kvim2:/dev/block # hexdump -n 520 /dev/block/mmcblk0boot0
0000000 0000 0000 0000 0000 0000 0000 0000 0000
*
0000200 9fa1 4f92 313d 34c4
0000208
kvim2:/dev/block # hexdump -n 520 /dev/block/mmcblk0boot1
0000000 0000 0000 0000 0000 0000 0000 0000 0000
*
0000200 9fa1 4f92 313d 34c4
0000208
kvim2:/dev/block # hexdump -n 520 /dev/block/mmcblk0
0000000 0000 0000 0000 0000 0000 0000 0000 0000
*
0000200 9fa1 4f92 313d 34c4
0000208
kvim2:/dev/block # hexdump -n 520 /dev/block/boot
boot bootloader
kvim2:/dev/block # hexdump -n 520 /dev/block/bootloader
0000000 0000 0000 0000 0000 0000 0000 0000 0000
*
0000200 9fa1 4f92 313d 34c4
0000208

you can also look around in /sys/class/mmc_host/emmc/emmc:0001/block/mmcblk0/
\end{verbatim}

this source code might also provide you some hints: 
\begin{verbatim}
https://github.com/khadas/linux/blob/khadas-vim-4.9.y/drivers/amlogic/mmc/emmc_partitions.c
https://github.com/khadas/u-boot/blob/ubuntu/common/store_interface.c#L1174

I don’t know if the FIP format has a human-readable specification, however you 
can take a look at fiptool’s source to see how it’s working: 
https://github.com/ARM-software/arm-trusted-firmware/tree/master/tools/fiptool 
https://github.com/ARM-software/arm-trusted-firmware/blob/master/include/tools_share/firmware_image_package.h

note also that some of the items in the amlogic boot images may be encrypted, 
so you may not read them directly (check the calls to aml_encrypt_$(SOC) in 
u-boot’s Makefile for that)
\end{verbatim}
