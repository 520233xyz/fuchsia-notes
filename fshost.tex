\section{fshost}

fshost是文件系统服务进程。

\begin{verbatim}
zircon/system/core/devmgr/fshost.c: main()

_fs_root = zx_get_startup_handle(PA_HND(PA_USER0, 0));
    拿到与devmgr通信通道
devfs_root = zx_get_startup_handle(PA_HND(PA_USER0, 1));
svc_root = zx_get_startup_handle(PA_HND(PA_USER0, 2));
zx_handle_t devmgr_loader = zx_get_startup_handle(PA_HND(PA_USER0, 3));
  这个是fshost为devmgr提供loader服务的handle
fshost_event = zx_get_startup_handle(PA_HND(PA_USER1, 0));

fshost_start();
    setup_bootfs();
        把/boot下的内容从bootdata里提出来
        setup_bootfs_vmo()
            system分区也是直接映射在内存里的vmo
            zx_vmar_map(zx_vmar_root_self(), 0, vmo, 0, size, ZX_VM_FLAG_PERM_READ, &address);
              映射vmo
            bootfs_create(&bfs, bootfs_vmo)
              这里把复制的bootfs vmo映射好
            bootfs_parse(&bfs, callback, &cd);

    vfs_global_init(vfs_create_global_root());
        vfs_create_global_root()
            创建memfs的根"<root>"
        memfs::global_vfs_root = root;
    fuchsia_start()
        zx_object_signal(fshost_event, 0, FSHOST_SIGNAL_READY);

vfs_connect_global_root_handle(_fs_root);
    vfs_connect_root_handle(memfs::global_vfs_root, h);
        vn->vfs()->ServeDirectory(fbl::RefPtr<fs::Vnode>(vn), fbl::move(ch));
            vn->Serve(this, fbl::move(channel), ZX_FS_RIGHT_ADMIN);
                vfs->ServeConnection(fbl::make_unique<Connection>(
                     vfs, fbl::WrapRefPtr(this), fbl::move(channel), flags));
                  connection->Serve()
                    wait_.set_object(channel_.get());
                    return wait_.Begin(vfs_->async());
                    vfs处理消息的入口实际上在Connection::HandleMessage(zxrio_msg_t* msg)


fs_root = fs_root_clone()
    vfs_create_global_root_handle(&h)
        zx::channel::create(0, &h1, &h2)
        vn->vfs()->ServeDirectory(fbl::RefPtr<fs::Vnode>(vn),
                                  fbl::move(h1))
        *out = h2.release()                         
    fs_root是通道的一头

fdio_ns_bind(ns, "/system", fs_clone("system"))
    zx_channel_create(0, &h0, &h1)
    fdio_open_at(fs_root, path, FS_DIR_FLAGS, h1)
        zxrio_connect(dir, h, ZXRIO_OPEN, flags, 0755, path)
            把h1发送给fs_root另一头的事件处理
    return h0

要创建loader service的服务端口了
loader_service_create_fs(NULL, &loader_service)
  loader_service_create_default(async, root_dir_fd, -1, fs_lib_paths,out);
    loader_service_create(async, &fd_ops, NULL, &svc);
      loader_service_addref(svc);
loader_service_attach(loader_service, devmgr_loader);

======
fs_clone("system")
当root vfs收到ZXRIO_OPEN之后:
vfs_是root_vfs
vnode_是global_root VnodeDir
比如path="system"

Connection::HandleMessage(zxrio_msg_t* msg)
  OpenAt(vfs_, vnode_, fbl::move(channel), fbl::StringPiece(path, len), flags, mode);
    vfs->Open(fbl::move(parent), &vnode, path, &path, open_flags, mode);
      OpenLocked(fbl::move(vndir), out, path, pathout, flags, mode);
        Vfs::Walk(vndir, &vndir, path, &path)
          这里返回的vndir是要打开的路径的父目录
        vfs_name_trim(path, &path, &must_be_dir)
        vfs_lookup(fbl::move(vndir), &vn, path);
          vn->Lookup(out, name);
            得到system目录对应的vnode
    vnode->Serve(vfs, fbl::move(channel), open_flags);
      vfs->ServeConnection(fbl::make_unique<Connection>(
        vfs, fbl::WrapRefPtr(this), fbl::move(channel), flags));





\end{verbatim}