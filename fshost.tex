\section{fshost}

fshost是文件系统服务进程。

\begin{verbatim}
zircon/system/core/devmgr/fshost.c: main()

_fs_root = zx_get_startup_handle(PA_HND(PA_USER0, 0));
    拿到与devmgr通信通道
devfs_root = zx_get_startup_handle(PA_HND(PA_USER0, 1));
svc_root = zx_get_startup_handle(PA_HND(PA_USER0, 2));
zx_handle_t devmgr_loader = zx_get_startup_handle(PA_HND(PA_USER0, 3));
fshost_event = zx_get_startup_handle(PA_HND(PA_USER1, 0));

fshost_start();
    setup_bootfs();
        把/boot下的内容从bootdata里提出来
        setup_bootfs_vmo()
            system分区也是直接映射在内存里的vmo? 有多大啊?


    vfs_global_init(vfs_create_global_root());
        vfs_create_global_root()
            创建memfs的根"<root>"
        memfs::global_vfs_root = root;
    fuchsia_start()
        zx_object_signal(fshost_event, 0, FSHOST_SIGNAL_READY);

vfs_connect_global_root_handle(_fs_root);
    vfs_connect_root_handle(memfs::global_vfs_root, h);
        vn->vfs()->ServeDirectory(fbl::RefPtr<fs::Vnode>(vn), fbl::move(ch));
            vn->Serve(this, fbl::move(channel), ZX_FS_RIGHT_ADMIN);
                vfs->ServeConnection(fbl::make_unique<Connection>(
                     vfs, fbl::WrapRefPtr(this), fbl::move(channel), flags));

                     wait_.set_object(channel_.get());
                     return wait_.Begin(vfs_->async());


fs_root = fs_root_clone()
    vfs_create_global_root_handle(&h)
        zx::channel::create(0, &h1, &h2)
        vn->vfs()->ServeDirectory(fbl::RefPtr<fs::Vnode>(vn),
                                  fbl::move(h1))
        *out = h2.release()                         
    fs_root是通道的一头

fdio_ns_bind(ns, "/system", fs_clone("system"))
    zx_channel_create(0, &h0, &h1)
    fdio_open_at(fs_root, path, FS_DIR_FLAGS, h1)
        zxrio_connect(dir, h, ZXRIO_OPEN, flags, 0755, path)
            把h1发送给fs_root另一头的事件处理
    return h0
\end{verbatim}