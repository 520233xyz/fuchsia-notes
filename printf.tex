\section{printf}

\begin{verbatim}


summary: 从efi获得framebuffer信息。用framebuffer模式输出信息。


dprintf()
#define dprintf(level, x...) do { if ((level) <= LK_DEBUGLEVEL) { printf(x); } } while (0)
  #define printf(x...) _printf(x)
    _printf_engine(__printf_output_func, NULL, fmt, ap);
      __kernel_stdout_write(s, len);
        __kernel_console_write(str, len);
          list_for_every_entry(&print_callbacks, cb, print_callback_t, entry) {
            if (cb->print)
                cb->print(cb, str, len);
          }

看看注册了哪些print callback
platform_early_display_init(void)
  info.framebuffer = (void*)X86_PHYS_TO_VIRT(bootloader.fb.base);
  bits是另外一块buffer
    // 从efi可以拿到framebuffer的base address
    gfxconsole_bind_display(&info, bits);
      gfx_init_surface_from_display(&hw, info)
        gfx_init_surface(surface, info->framebuffer, info->width, info->height, info->stride, info->format, flags);
      gfxconsole_setup(s, &hw_surface);
      register_print_callback(&cb);

        cb = gfxconsole_print_callback

\end{verbatim}