\section{devmgr}

\begin{verbatim}
进程和通道handle的关系
devmgr:appmgr_req_cli <--> appmgr:appmgr_req_srv
devmgr:svchost_outgoing <--> svchost:dir_request
devmgr:fs_root <--> fshost:handles[0]
svchost:appmgr_svc <--> appmgr:appmgr_svc_req

devmgr.c: main()

先把从userboot传过来的loader service handle关掉

devmgr_io_init();
    zx_log_create(0, &h)
    logger = fdio_logger_create(h)
    fdio_bind_to_fd(logger, 1, 0);
        fdio_fdtab[fd] = io;

root_resource_handle = zx_get_startup_handle(PA_HND(PA_RESOURCE, 0));
    这个要追溯到zircon/kernel/lib/userboot/userboot.cpp:
    get_resource_handle(&handles[BOOTSTRAP_RESOURCE_ROOT]);
        ResourceDispatcher::Create(&root, &rights, ZX_RSRC_KIND_ROOT, 0, 0);
        这个root resource handle好像只是在关机时校验一下

devfs_init(root_job_handler)
    zx_channel_create(0, &h0, &h1)
    iostate_create(&root_devnode, h0)
    devfs_root = h1;
    devfs_root是channel的一端,可以向另一端dc_rio_handler事件处理循环发消息


zx_channel_create(0, &appmgr_req_cli, &appmgr_req_srv);
    建立跟appmgr的通信通道, appmgr_req_srv会发送给appmgr

zx_event_create(0, &fshost_event);
    fs_host准备好的事件

devmgr_svc_init()
    svchost_start()
        zx_channel_create(0, &dir_request, &svchost_outgoing)
        zx_channel_create(0, &appmgr_svc_req, &appmgr_svc)

        fdio_service_connect_at(appmgr_req_cli, "svc", appmgr_svc_req)
            zxrio_connect(dir, h, ZXRIO_OPEN, ZX_FS_RIGHT_READABLE |
                          ZX_FS_RIGHT_WRITABLE, 0755, path)
                把appmgr_svc_req发送给appmgr_req_cli的另一端appmgr_req_srv
                appmgr_svc后面会发给svchost,相当于建立了appmgr和svchost之间的通信通道
        launchpad_add_handle(lp, dir_request, PA_DIRECTORY_REQUEST);
            svchost_outgoing通向svchost进程,后面fs_clone("svc")会用到
        launchpad_add_handle(lp, appmgr_svc, PA_HND(PA_USER0, 0));
        launchpad_go(lp, &process, &errmsg)
            启动svchost进程

devmgr_vfs_init()
    fshost_start()
        zx_channel_create(0, &fs_root, &handles[0])
            fs_root通道通往fshost的/
        handles[n] = devfs_root_clone()
            fdio_service_clone(devfs_root)
                zx_channel_create(0, &cli, &srv)
                zxrio_connect(svc, srv, ZXRIO_CLONE, ZX_FS_RIGHT_READABLE |
                              ZX_FS_RIGHT_WRITABLE, 0755, "")
                    把srv发送给devfs_root的另一端,也就是dc_rio_handler事件处理循环
                    对ZXRIO_CLONE的处理是devfs_open(h) 设备节点是root
                    把这个handle又挂在了dc_rio_handler事件处理循环上,挂在dc_port上
                return cli
                也就是建立了fshost与devfs的通信通道

        handles[n] = fs_clone("svc")
            zx_channel_create(0, &h0, &h1)
            fs = svchost_outgoing
            fdio_open_at(fs, path, flags, h1)
                zxrio_connect(dir, h, ZXRIO_OPEN, flags, 0755, path)
                    把h1发送给svchost
            return h0
            建立fshost与svchost的通信通道

        zx_channel_create(0, &ldsvc, &handles[n])
            让ldsvc通往fshost,看来fshost要承担elf加载的任务?

        zx_handle_duplicate(bootfs.vmo, ZX_RIGHT_SAME_RIGHTS, &handles[n])
            让fshost能够访问boot fs的vmo

        zx_handle_duplicate(fshost_event, ZX_RIGHT_SAME_RIGHTS, &handles[n])
            让fshost完事之后通过fshost_event通知devmgr

        handles[n] = zx_get_startup_handle(PA_HND(PA_VMO_BOOTDATA, m))
            让fshost能访问bootdata. bootdata包含压缩的boot fs
        
        把vdso vmo给fshost

        devmgr_launch(svcs_job_handle, "fshost", argc, argv,
                      envp, -1, handles, types, n, NULL, 0);
            启动fshost

    fdio_ns_create(&ns)
        fdio_ns_bind(ns, "/system", fs_clone("system"))
            zx_channel_create(0, &h0, &h1)
            fdio_open_at(fs, path, flags, h1)
                fs通向fshost
                zxrio_connect(dir, h, ZXRIO_OPEN, flags, 0755, path)
            return h0


thrd_create_with_name(&t, service_starter, NULL, "service-starter")


service_starter(void* arg)
    thrd_create_with_name(&t, fuchsia_starter, NULL, "fuchsia-starter")

fuchsia_starter(void* arg)
    zx_object_wait_one(fshost_event, FSHOST_SIGNAL_READY, deadline, NULL)
    zx_object_signal(fshost_event, FSHOST_SIGNAL_READY, 0)
        再signal一次,目前没有人等待这个信号
    把appmgr_req_srv发送给appmgr,建立devmgr和appmgr之间的通道
    devmgr_launch(fuchsia_job_handle, "appmgr", countof(argv_appmgr),
                          argv_appmgr, NULL, -1, appmgr_hnds, appmgr_ids,
                          appmgr_hnd_count, NULL, FS_FOR_APPMGR)
        会把FSTAB里除了FS_HUB(因为是appmgr自己)对应的fs_clone出来的handle都给appmgr发过去
            包括devfs, fshost, svchost

    

\end{verbatim}


有一个中心服务devmgr coordinator. 它会扫描驱动,为每个驱动launch一个devhost进程。launch完了之后会发送
create device, bind driver这2个rpc给devhost. devhost会处理这2个rpc. 对于root device,不会处理,
因为libname=""

文档推荐在bind driver里调用add device。驱动本身是在devhost里运行的。devhost会rpc调用dev coordinator
的add device方法。这个方法主要是创建dc里对设备的代表,发布到devfs里。


加载驱动的过程:

dc进程:
\begin{verbatim}
coordinator()
  find_loadable_drivers()
    di_read_driver_info()
      found_driver()
        dc_driver_added()
          if dc_running  then queue WORK_DRIVER_ADDED  实际上dc_running == false

process_work()
WORK_DRIVER_ADDED:
  dc_handle_new_driver()
    dc_bind_driver()
      dc_attempt_bind()
        dc_prepare_proxy()
          dc_create_proxy()  为父设备创建一个代理设备
          创建h0, h1
          dc_new_devhost()   proxy->devhost会与devhost进程有channel连接
          dh_create_device()  向devhost进程发送DC_OP_CREATE_DEVICE消息,
                              创建新的channel. 一头设置给proxy dev,在dc里监听起来。
                              另一头作为handle[0]传给devhost(存放在devhost里proxy device里面,
                              作为其他设备的parent).  从而dc和dh里面的proxy device之间
                              建立了rpc通道。


                              h1作为handle[2]传过去,这个目前似乎只有pci_rpc_request用于
                              发布出去的rpcch,
                              跟别的关系不大。

          dh_connect_proxy()  把h0作为DC_OP_CONNECT_PROXY传过去
        dh_bind_driver()  向dh进程发送DC_OP_BIND_DRIVER消息。

DC_OP_ADD_DEVICE:
  dc_add_device()
    创建子device, 把发来的hrpc设置好。监听。
    添加WORK_DEVICE_ADDED

WORK_DEVICE_ADDED:
  dc_handle_new_device()
    寻找匹配驱动
    dc_attempt_bind()
                
dh进程:
DC_OP_CREATE_DEVICE:
  dh_find_driver()
  driver->ops->create(hin[2])
    device_add()
      device_add_from_driver()
        devhost_device_create()
        devhost_device_add()
          proxy device case: 把rpc记录到dev里面,返回
          
  监听hin[0], 至此proxydev和 devhost之间建立了rpc通道。

DC_OP_CONNECT_PROXY:
  proxy_ios_create()  在devhost进程里监听传来的h0

DC_OP_BIND_DRIVER:
  调用driver的bind()

driver's bind op:
  device_add()
    device_add_from_driver()
      devhost_device_add()
        devhost_add()
          创建hrpc, hsend
          给dc发送DC_OP_ADD_DEVICE消息,把hsend发回去,监听hrpc.

coordinator()
首先创建sys, test二个devhost进程。
在遍历驱动的时候,创建root, misc(console)二个devhost 进程

dc prepare proxy引起的devhost的驱动的create方法
第一次调用add device只会把hrpc加入
devhost端的proxy设备里。之后的驱动的create方法中的add device才会
引起dc端的消息。
在dc里,设备都是创建在proxy device下面的。

dmctl可以动态添加driver.

例子:
dc: create device "platform"
dh: add device platform
dc: 匹配到qemu-bus 驱动,让dh那边加载这个驱动
dh: 调用了qemu-bus驱动,里面又有Add device, 回到dc
dc: 创建pci, pl031设备



qemu_bus_bind()
    device_get_protocol(parent)
        dev->ops->get_protocol(dev->ctx == dev)
          platform_bus_get_protocol()
              protocol->ctx= bus = ctx 就是dev
\end{verbatim}

devmgr最后会创建一个新的线程启动\verb|service_starter()|
\begin{verbatim}
service_start()
        thrd_create_with_name(&t, analyzer_starter, NULL, "analyzer-starter")
        如果没有禁止netsvc, devmgr_launch("/boot/bin/netsvc")
        如果没有禁止virtcon, devmgr_launch("/boot/bin/virtual-console")
        至此,zircon layer里的活应该干完了。
        thrd_create_with_name(&t, fuchsia_starter, NULL, "fuchsia-starter")
\end{verbatim}

\begin{verbatim}
fuchsia_start()
        wait for fshost to start
        
        load_system_drivers()
                find_loadable_drivers("/system/driver");
                find_loadable_drivers("/system/lib/driver");

        devmgr_launch(fuchsia_job_handle, "appmgr", countof(argv_appmgr),
                          argv_appmgr, NULL, -1, appmgr_hnds, appmgr_ids,
                          appmgr_hnd_count, NULL, FS_FOR_APPMGR);

\end{verbatim}

应用打开设备:发送一个channel handle给devfs, devmgr将handle转发给devhost, 
devhost将handle挂在自己的port上。
